\documentclass[12pt,a4paper]{article}
\begin{document}
\title{The Effect of high population growth rate in Uganda}
\author{OCAN JOHN JOEL.\\
COURSE NAME: RESEARCH METHODOLOGY\\
   COURSE CODE:BIT2207\\
    REG NUMBER: 16/U/10705/ps.\\
      STUDENT NUMBER: 216009506\\}
\maketitle   
\section{The effects of high population growth rate inUganda}
    The total population in Uganda was estimated at 41.5 million people in 2016, according to the latest census figures. Looking back, in the year of 1960, Uganda had a population of 6.8 million people which implies that the Uganda population is growing at a faster rate. \\
    Basic research was carried out aiming at increasing scientific knowledge and the effect of the growing population on Uganda's resources like cutting down of trees.\\
    The government of Uganda is using applied research to solve the effects of increasing population growth rate on the resources by sensitizing her people to use good family methods to reduce on the over whelming population growth rate.\\
    
    Descriptive research was achieved through the use of questionaires to find out the number of people in a particular family or clan by designing some questions. The analytical research was achieved by the facts or population census carried out in the previous population census like for example in 1960.\\
Qualitative research was carried out by expressing in form of graphs or charts the number of people against the year to give a meaningful analysis of the growing population while the quantitative research involved the use of some variables whose values were expressed in numerical forms.
    
    
    
    
    
    
\end{document}

